\documentclass{article}

\usepackage[a4paper]{geometry}        % using A4 paper
\usepackage[utf8]{inputenc}           % danish characters without codes
\usepackage{xcolor}                   % define own named colors
\usepackage{graphicx}                 % includegraphics
\usepackage{graphics}                 % scalebox
\usepackage{hyperref}                 % hyperlinks
\usepackage{listings}                 % standard code inclusion
\usepackage[nottoc]{tocbibind}        % bibliography,LoF in ToC

\newcommand{\textdesc}[1]{\textit{\textbf{#1}}}
\newcommand{\descitem}[1]{\item \textdesc{#1}}
\newcommand{\includeSVG}[1]{
  \includegraphics[scale=1.0]{./figs/#1.pdf}
}

% nicer form of inter-paragraph layout
\setlength{\parindent}{0in}
\setlength{\parskip}{3mm}

\title{Post-Meeting Notes\\\scalebox{.85}{from 23rd Mar, 2016}}
\author{Aslak Johansen \href{mailto:asjo@mmmi.sdu.dk}{asjo@mmmi.sdu.dk}}

\begin{document}

\maketitle
\tableofcontents
\newpage

\section{Introduction}

% base graph (intention, representing relationships as graph, graph type aspects, definition), 
We want to represent all the relationships that we need to model as a multi-aspect graph. That is, a set of graphs -- each representing an aspect -- whereof a given node may be present in any subset. The combined graph is referred to as \textsl{the base graph}.

% aspects
Relevant aspects include (but are not limited to):
\begin{itemize}
  \descitem{Physical graph} This has the notion of spaces (aka rooms), boundaries, floors and floors. Room adjacency and room connectivity is modeled.
  \descitem{HVAC graph} This includes (i) flows of air, steam and fluids, (ii) control signals and logic configuration, and (iii) HVAC zones. There is a many-to-many relationship between HVAC zones and spaces.
  \descitem{Light graph} This includes (i) flows of light (windows and other optical constructs), (ii) control signals and logic configuration, and (iii) light zones. There is a many-to-many relationship between light zones and spaces.
  \descitem{Electrical graph} Usually a distribution tree of meters, fuses and loads. However, it will become more complex in case of on-site renewables.
  \descitem{Water graph} May be split into clean, gray and brown water?
\end{itemize}

% utility graphs (derived, convenient for resolving queries, view equivalent)
From the base graph several utility graphs may be derived as convenient aides in query resolution. These are the functional equivalent of views. We refer to them as \textsl{derived graphs}.

\section{Base Graph}

An edge:
\begin{itemize}
  \descitem{Typed} An edge has a type from a semi-fixed set\footnote{This involves effort being spent finding the right \textsl{small} set that covers our use-cases while allowing for extensions}.
\end{itemize}

A node:
\begin{itemize}
  \descitem{Attributes} A node has a key-value store.
  \descitem{Level} A node is one of:
    \begin{itemize}
      \descitem{Primitive}
      \descitem{Complex}
    \end{itemize}
\end{itemize}

\section{Derived Graphs}

At the moment the main concerns are the base graph and the ability to run queries on it. With that in place it should be possible to create derived graphs.

\section{The AHU-1 Example}

Figure \ref{fig:ahu1} illustrates the graph we came up with for the flow, contains and controls relations.

I have noticed two issues:
\begin{enumerate}
  \item In this example "Fan" is the name of a primitive node and as well as a complex node. In future naming, we should try to avoid this. It leads to awkwardly long fully-qualified names as illustrated in the following issue:
  \item The relationship between the pressure controller and the primitive fan node crosses the boundary of the complex fan through a port. It is thus modeled as two edges, both named 'o' according to their output role to the controller. However, from perspective of the complex fan it is most definitely an input. As we need the ability to reason about partial graphs, perhaps we should find a side-independent name for both controller input and output.
\end{enumerate}

\begin{figure}[tp]
  \begin{center}
    \scalebox{0.8}{\includeSVG{ahu1_from_writeboard}}
  \end{center}
  \caption{The flow, contains and control relations of Soda Hall's AHU-1.}
  \label{fig:ahu1}
\end{figure}

\section{Prototype}

Gabe mentioned networkx. I will look into that and -- if it is a nice fit -- start writing code.

\end{document}

